\documentclass{article}
\usepackage[utf8]{inputenc}
\usepackage{amssymb}
\usepackage{graphicx}

\title{Trabalho 1 - Teoria da Computação}
\author{Gabriel Leal C. Amaral, Gustavo F. Olegário}
\date{October 2016}

\begin{document}
\maketitle
\section{Máquina 1a - Fita única}
    \subsection{Enunciado da linguagem}
        \text{L = \{w\#w / w \in{\mbox{\{a,b,c\}* }}\}}
    \subsection{Descrição do algoritmo}
        \begin{enumerate}
        \item No estado inicial, se ler a entrada vazia, aceite a palavra. Caso contrário, vá para o próximo estado.
        \item Marcar a posição lida com x caso seja a ou b.
        \item Percorrer até o primeiro elemento após a cerquilha que seja diferente de x.
        \item Caso esse elemento seja diferente do último marcado com x no lado esquerdo da cerquilha, rejeite. Caso contrário, marque com x e volte até o último elemento marcado no lado esquerdo da cerquilha. Se o elemento seguinte for x, aceite a palavra. Caso seja a ou b, vá para 2. Caso contrário, rejeite.
        \end{enumerate}
        \text{As imagens dos testes estão em /images/1a}
        \text{A codificação da máquina está em machines\_jflap}
\section{Máquina 2a - Multifita}
    \subsection{Enunciado da lingaguem}
        \text{L = \{ ww / w \in{\mbox{\{0,1\}* }}\}}
    \subsection{Descrição do algoritmo}
        \begin{enumerate}
        \item No estado inicial, existem três possibilidades. Se for uma 
        palavra do alfabeto inicial, 0 ou 1, marca como lida com X ou Y, 
        respectivamente. Caso contrário, segue a sequência de estados para
        o estado final
        \item Se no estado inicial, a máquina tiver lido 1 ou 0, a máquina
        prosseguirá até encontrar uma palavra vazia.
        \item Ao encontrar uma palavra vazia, irã retroceder até encontrar 1 ou 0.
        \item Se encontrar um 1 ou 0, marcará como lido e irá retroceder até
        encontrar uma palavra não lida. Quando achar que for lida, avança
        uma casa a frente.
        \item Caso tenha seguido o caminho para o estado final, para 
        cada elemento já lido na fita 1, marcará com um símbolo diferente 
        e escreverá o antigo símbolo na fita 2.
        \item Por último, percorrerá a ambas as fitas até a extremidade 
        esquerda, garantindo que as duas fitas tem a mesma palavra.
        \end{enumerate}
        \includegraphics[scale=0.3]{Maq2a}
\section{Máquina 2b - Multifita}
    \subsection{Enunciado da linguagem}
        \text{L = \{ w$w^{R}$w / w \in{\mbox{\{a,b\}* }}\}}
    \subsection{Descrição do algoritmo}
        \begin{enumerate}
        \item No estado inicial, se ler a entrada vazia, aceite a palavra. Caso contrário, vá para o próximo estado.
        \item Copie cada entrada lida da fita 1 para a fita 2. Ao chegar na entrada vazia, volte uma posição à esquerda em ambos os cabeçotes.
        \item Mova o cabeçote da fita 1 para esquerda. Mova o cabeçote da fita 2, 3 vezes para a esquerda.
        \item Repita o passo anterior até o cabeçote 2 chegar à entrada vazia. Chegando na entrada vazia, mova o cabeçote das fitas 1 e 2 para a direita. Neste ponto, o cabeçote 1 aponta para o início da última string e o cabeçote dois para o começo da primeira.
        \item Percorrer até o final da string analisada na fita 1. Caso as entradas sejam iguais, marcar com x ou y (a ou b). Caso contrário, rejeite. Continue até a entrada vazia.
        \item Mova o cabeçote uma posição à esquerda na fita 1 (neste momento, o cabeçote da fita 1 irá apontar para o final da última string) e mantenha o cabeçote 2 onde está (neste ponto, ele está no começo da string reversa).
        \item Percorra a fita 1 no sentido direita-esquerda e a fita 2 no sentido esquerda-direita. Caso as entradas sejam iguais, continue até atingir uma entrada a ou b na fita 1 (alertando que acabou a string). Atingindo esta marca, aceite a palavra.
        \end{enumerate}
        \text{As imagens dos testes estão em images/2b}
        \text{A codificação da máquina está em machines\_jflap}
\section{Máquina 3a - Blocos}
    \subsection{Enunciado da linguagem}
        \text{L = \{ $0^{2}^{n}$ / n \ge 0 \}}
    \subsection{Descrição do algoritmo}
    \begin{enumerate}
        \item Inicialmente, verifica-se se a palavra é vazia ou não.
        Se for, vai para o estado de rejeição. Caso contrário se for
        um 0, escreve x e vai para o próximo estado.
        \item No próximo estado, ficará procurando por 0 ou por palavra
        vazia. Caso seja uma palavra vazia, significa que entrou num estado
        de aceitação, então aceita. Se encontrar x apenas avança na fita
        e se encontrar um 0 escreve x na fita e vai para o próximo estado.
        \item Ficará nesse estado até encontrar um a palavra vazia ou 
        um 0. Em caso de x, apenas avança na fita. Se encontrar uma palavra
        vazia, retrocede na fita e troca de estado. Caso encontre um 0,
        avança na fita e vai pro outro estado.
        \item Se encontrar um 0, escreve x na fita e volta para o estado an
        terior. Se encontrar um x, apenas avança na fita e caso encontre
        uma palavra vazia significa que a palavra ao todo tem um número
        diferente de $0^{2}^{n}$, então vai para o estado de rejeição.
        \item Ficará voltando na fita, para cada x e 0 na fita. Quando
        encontrar uma palavra vazia, avança e volta para o segundo 
        estado da máquina
    \end{enumerate}
\end{document}
